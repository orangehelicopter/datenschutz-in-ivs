%% LaTeX2e class for seminar theses
%% sections/conclusion.tex
%% 
%% Karlsruhe Institute of Technology
%% Institute for Program Structures and Data Organization
%% Chair for Software Design and Quality (SDQ)
%%
%% Dr.-Ing. Erik Burger
%% burger@kit.edu
%%
%% Version 1.0.2, 2020-05-07

\section{Zusammenfassung und Ausblick}
\label{ch:Conclusion}

Im folgenden Abschnitt wird der Inhalt und das Ziel der Seminararbeit zusammengefasst. Abschließend werden die Erkenntnisse diskutiert und ein Ausblick über die zukünftige Entwicklung gegeben. 

\subsection{Zusammenfassung}

Das Ziel von der Seminararbeit war es, die C2X Kommunikation von technischer und datenschutzrechtlicher Sicht zu untersuchen. Dazu wird zunächst im Kapitel \ref{ch:Background} auf den Hintergrund der C2X Kommunikation eingegangen. Der Begriff von Intelligenten Verkehrssystemen (IVS) wird eingeführt und erklärt, und es werden einige bereits existierende Initiativen zur Einführung der IVS genannt, wie z.B. die Rechtsgrundlage IVS-RL und das Car-to-Car Communication Consortium.

Weiterhin wird im Kapitel \ref{ch:FirstContentSection} die technische Funktionsweise der C2X Kommunikation untersucht. Zuerst wird ein Überblick über die WAVE Architektur gegeben, die für IVS heutzutage verwendet wird, und den IEEE 802.11p WLAN-Funkstandard, auf dem die C2C Kommunikation basiert. Ferner wird die Funktionsweise der Public-Key Infrastruktur (PKI) erläutert, die den sicheren Nachrichtenaustausch zwischen End-Entitäten (Fahrzeugen und Infrastrukturkomponenten) ermöglicht. Zusätzlich werden die Besonderheiten der PKI-Implementierung in Europa erläutert. Zum Schluss werden die Nachrichtenformate aufgezählt und beschrieben, die für aktuelle Nutzungsfälle im IVS Kontext verwendet werden. 

Im Kapitel \ref{ch:SecondContentSection} wird geprüft, ob die DSGVO auf die Fahrdaten im Kontext von IVS anwendbar ist. Dafür werden zuerst Methoden angeführt, wie ein Angreifer Fahrdaten aufzeichnen kann, um einzelne Bewegungsprofile zu erhalten. Darüber hinaus werden die Möglichkeiten diskutiert, Verhaltensprofile aus Bewegungsprofilen abzuleiten, und deren Personenbezug herzustellen. 

Da die Anwendbarkeit der DSGVO im Kapitel \ref{ch:SecondContentSection} bestätigt wurde, betrachten wir im Kapitel \ref{ch:ThirdContentSection} die Vereinbarkeit mit relevanten Grundsätzen der DSGVO. Für jeden angeführten Grundsatz wird dessen Einhaltung geprüft und datenschutzrechtlich relevante Punkte erwähnt, die bei der Implementierung einer IVS Infrastruktur beachtet werden müssen.


\subsection{Erkenntnisse und Ausblick}

