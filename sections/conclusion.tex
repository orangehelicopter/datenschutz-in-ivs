%% LaTeX2e class for seminar theses
%% sections/conclusion.tex
%% 
%% Karlsruhe Institute of Technology
%% Institute for Program Structures and Data Organization
%% Chair for Software Design and Quality (SDQ)
%%
%% Dr.-Ing. Erik Burger
%% burger@kit.edu
%%
%% Version 1.0.2, 2020-05-07

\section{Zusammenfassung und Ausblick}
\label{ch:Conclusion}

Im folgenden Abschnitt wird der Inhalt und das Ziel der Seminararbeit zusammengefasst. Abschließend werden die Erkenntnisse diskutiert und ein Ausblick über die zukünftige Entwicklung gegeben. 

\subsection{Zusammenfassung}

Das Ziel von der Seminararbeit war es, die C2X Kommunikation von technischer und datenschutzrechtlicher Sicht zu untersuchen. Dazu wurde zunächst im Kapitel \ref{ch:Background} auf den Hintergrund der C2X Kommunikation eingegangen. Der Begriff von Intelligenten Verkehrssystemen (IVS) wurde eingeführt und erklärt, und es wurden einige bereits existierende Initiativen zur Einführung der IVS genannt, wie z.B. die Rechtsgrundlage IVS-RL und das Car-to-Car Communication Consortium.

Weiterhin wurde im Kapitel \ref{ch:FirstContentSection} die technische Funktionsweise der C2X Kommunikation untersucht. Zuerst wurde ein Überblick über die WAVE Architektur gegeben, die für IVS heutzutage verwendet wird, und den IEEE 802.11p WLAN-Funkstandard, auf dem die C2C Kommunikation basiert. Ferner wurde die Funktionsweise der Public-Key Infrastruktur (PKI) erläutert, die den sicheren Nachrichtenaustausch zwischen End-Entitäten (Fahrzeugen und Infrastrukturkomponenten) ermöglicht. Zusätzlich wurde die Besonderheiten der PKI-Implementierung in Europa erläutert. Zum Schluss wurden die Nachrichtenformate aufgezählt und beschrieben, die für aktuelle Nutzungsfälle im IVS Kontext verwendet werden. 

Im Kapitel \ref{ch:SecondContentSection} wurde geprüft, ob die DSGVO auf die Fahrdaten im Kontext von IVS anwendbar ist. Dafür wurden zuerst Methoden angeführt, wie ein Angreifer Fahrdaten aufzeichnen kann, um einzelne Bewegungsprofile zu erhalten. Darüber hinaus wurden die Möglichkeiten diskutiert, Verhaltensprofile aus Bewegungsprofilen abzuleiten, und deren Personenbezug herzustellen. 

Da die Anwendbarkeit der DSGVO im Kapitel \ref{ch:SecondContentSection} bestätigt wurde, betrachten wir im Kapitel \ref{ch:ThirdContentSection} die Vereinbarkeit mit relevanten Grundsätzen der DSGVO. Für jeden angeführten Grundsatz wurde dessen Einhaltung geprüft und datenschutzrechtlich relevante Punkte erwähnt, die bei der Implementierung einer IVS Infrastruktur beachtet werden müssen.


\subsection{Erkenntnisse und Ausblick}

Unser Alltag wird immer mehr von neuen Technologien geprägt, und Straßenverkehr ist keine Ausnahme. Intelligente Verkehrssysteme (IVS) sollen nach und nach für die Massen eingeführt werden, um die Verkehrssicherheit zu verbessern und eine Plattform für neue technologische Entwicklung bereitzustellen. Allerdings bedarf die Einführung der IVS nicht nur neue technische Lösungen, sondern auch gesetzliche Rahmen, die eingehalten werden müssen. 

Einer der wichtigsten Gründe für diese Anforderung ist die aktuelle Möglichkeit für flächendeckende Erhebung der CAMs über eine längere Zeit. Dies kann weitgehende Eingriffe in die Privatsphäre der Einzelpersonen und allgemein in das informationelle Selbstbestimmungsrecht verursachen. Es bedarf deswegen eine gesetzliche Regelung und kann nicht allein durch eine Zustimmung legitimiert werden\footnote{\emph{Kiometzis/Ullmann}, DuD 2017, S. 231.}. Darüber hinaus könnten technische Lösungen implementiert werden, die eine Weiterverwendung von CAMs verhindern, z.B. durch die Gewährleistung der Flüchtigkeit von darin enthaltenen Daten. Da die CAMs unverschlüsselt versenden werden, müssen andere Wege gefunden werden, die persönliche Information vor Missbrauch zu schützen.

Der Versand von CAMs ist ein wichtiger Teil der Digitalisierung im Straßenverkehr. Er soll dazu beitragen, Unfälle zu vermeiden und somit die Verkehrssicherheit zu steigern. Allerdings entsteht aktuell ein Risiko für die Privatsphäre durch die Verlinkbarkeit der einzelnen CAMs zu CAM Traces und die Nichtanfechtbarkeit des auf dieser weise aufgezeichneten Fahrwegs. Diese Fragestellungen sollten im Mittelpunkt der weiteren Entwicklungen stehen - z.B. ob es Alternativen zu elektronischen Signaturen gibt, die keine Nichtanfechtbarkeit garantieren. Auch könnte ein selektiver Ansatz für den Versand von CAMs in Erwägung gezogen werden, um die Frequenz und somit die Verlinkbarkeit von einzelnen Nachrichten zu reduzieren. Von Kiometzis u.a. wird es vorgeschlagen, den Versand von CAMs nur auf die bekannten Unfallbrennpunkte oder z.B. Kreuzungen zu reduzieren.

Darüber hinaus sollte es in Betracht gezogen werden, ob CAMs für alle Nutzungsfälle angemessen wären, und nicht durch andere Nachrichtenformate ersetzt werden sollten. Die meisten darin übertragenen Daten sind für spätere kooperative Szenarien vorgesehen, daher wären sie eventuell in separaten Nachrichtenformaten besser aufgehoben. 

Abschließend kann man die Wichtigkeit der Einhaltung von datenschutzrechtlichen Prinzipien beim Entwurf einer IVS Infrastruktur in Europa unterstreichen. Die DSGVO gilt europaweit und umfasst mehrere Grundsätze zum Schutz personenbezogenen Daten. Bei der Nichteinhaltung der in dieser Seminararbeit erwähnten Grundsätze werden nach Art. 83 Abs. 5 DSGVO ``Geldbußen von bis zu 20 000 000 EUR oder im Fall eines Unternehmens von bis zu 4 \% seines gesamten weltweit erzielten Jahresumsatzes des vorangegangenen Geschäftsjahrs verhängt, je nachdem, welcher der Beträge höher ist". Deswegen ist eine gründliche datenschutzrechtliche Betrachtung im Interesse der Verantwortlichen. Die Bedeutung der Datenschutz für den einzelnen Nutzer ist ebenfalls nicht zu unterschätzen, da es sich unmittelbar um seine Freiheiten handelt. Daher ist es notwendig, in dem neu entstehenden Kontext von IVS rechtliche und technische Aspekte des Datenschutzes zu erarbeiten und so eine sichere und rechtmäßige Lösung zu erreichen.






