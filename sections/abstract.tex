%% LaTeX2e class for seminar theses
%% sections/abstract_en.tex
%% 
%% Karlsruhe Institute of Technology
%% Institute for Program Structures and Data Organization
%% Chair for Software Design and Quality (SDQ)
%%
%% Dr.-Ing. Erik Burger
%% burger@kit.edu
%%
%% Version 1.0.2, 2020-05-07


Der Einsatz der Intelligenten Verkehrssysteme (IVS) wird in kommenden Jahren den Verkehrsalltag stark beeinflussen. C2X-Kommunikation (Car to X) beschreibt die Verbindung von Fahrzeugen untereinander und der verkehrssteuernden Infrastruktur - diese Technologie hat das Potenzial, zur Straßensicherheit und verbessertem Verkehrsfluss wesentlich beizutragen. Dafür müssen große Datenmengen gesammelt und übertragen werden, was möglicherweise datenschutzrechtliche Fragen mit sich bringt.

Der folgende Beitrag stellt die technische Funktionsweise der C2X-Kommuni-kation vor und analysiert deren datenschutzrechtliche Relevanz hinsichtlich der DSGVO. Es wird gezeigt, dass es sich bei den für die C2X-Kommunikation versendeten Nachrichten um personenbezogene Daten handelt, und im Anschluss wird die Vereinbarkeit mit datenschutzrechtlichen Prinzipien auf Basis der DSGVO diskutiert.