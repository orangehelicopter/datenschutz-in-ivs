%% LaTeX2e class for seminar theses
%% sections/content.tex
%% 
%% Karlsruhe Institute of Technology
%% Institute for Program Structures and Data Organization
%% Chair for Software Design and Quality (SDQ)
%%
%% Dr.-Ing. Erik Burger
%% burger@kit.edu
%%
%% Version 1.0.2, 2020-05-07

\section{Einleitung}
\label{ch:Introduction}

%% -------------------
%% | Example content |
%% -------------------

Motivation

In näher Zukunft werden nach und nach im Alltag Fahrzeuge eingeführt, die miteinander (Car-to-Car, kurz C2C) und mit der Infrastruktur (Car-to-Infrastructure, kurz C2I) kommunizieren können. Der Oberbegriff zu dieser Art der Kommunikation lautet Car-to-X (C2X). Der Datenaustausch in solchen Kooperativen Intelligenten Systemen (C-ITS) findet statt, um Unfälle auf den Straßen zu vermeiden und damit die Sicherheit im Straßenverkehr zu gewährleisten. Dafür senden Fahrzeuge gegenseitig verkehrsrelevante Daten, wie zum Beispiel Beschleunigung, Geschwindigkeit, Länge und Gewicht des Fahrzeugs. Diese Daten werden verwendet, um ein Verkehrslagebild zu erstellen und zu verteilen, damit teilnehmende Fahrzeuge stets die aktuellsten Informationen besitzen und auf eintretende Verkehrssituationen geeignet reagieren können. 

...\cite{Strubbe2017}

Neue Problemstellung, wozu braucht man Datenschutz und Privacy? Was ist das Ziel der Seminararbeit?

Integrität und Vertraulichkeit!




\section{Hintergrund}
\label{ch:Citation}

IVS-RL unbedingt einführen!!!

Begriff C-ITS - \cite{CITS2016}, andere Begriffe einführen.

Initiativen in Deutschland und EU, Geschichte und Entwicklung

Car-2-Car Communication Consortium - \cite{Car2Car}

Related work - Literatur, die sich mit der Fragenstellung beschäftigt - z.B. \cite{Jochum2020}

Die EU hat bereits vor zehn Jahren eine Rechtsgrundlage für intelligente Verkehrssysteme geschaf- fen. Es handelt sich um die RL 2010/40/EU v. 7.7.2010 zum Rahmen für die Einführung intelligenter Verkehrssysteme im Straßenverkehr und für deren Schnittstellen zu anderen Verkehrsträgern (IVS-
2
RL). 

Die Bundesrepublik Deutschland hat hinsichtlich der Umsetzung der IVS-RL und der Ausfüllung der delegierten Verordnungen die föderale Struktur zu beachten. Die Umsetzung der Richtlinie erfolgte
17 18
2013 durch das Gesetz über Intelligente Verkehrssysteme im Straßenverkehr , welches 2017 geändert wurde, um den Anforderungen der delegierten Verordnungen zu genügen.
- Jochum: Verkehrsdaten für intelligente Verkehrssysteme ZD 2020, 497

%A citation: \cite{becker2008a} For referencing, see \autoref{sec:Introduction:Figures}

%\subsection{Example: Figures}
%\label{sec:Introduction:Figures}
%\begin{figure}
%\centering
%\includegraphics[width=4cm]{images/sdqlogo}
%\caption{SDQ logo}
%\label{fig:sdqlogo}
%\end{figure}

%A reference: The SDQ logo is displayed in \autoref{fig:sdqlogo}. 
%(Use \code{\textbackslash autoref\{\}} for easy referencing.) 
%
%\subsection{Example: Tables}
%\label{sec:Introduction:Tables}
%\begin{table}
%\centering
%\begin{tabular}{r l}
%\toprule
%abc & def\\
%ghi & jkl\\
%\midrule
%123 & 456\\
%789 & 0AB\\
%\bottomrule
%\end{tabular}
%\caption{A table}
%\label{tab:atable}
%\end{table}

%% --------------------
%% | /Example content |
%% --------------------