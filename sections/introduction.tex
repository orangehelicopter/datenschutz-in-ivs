%% LaTeX2e class for seminar theses
%% sections/content.tex
%% 
%% Karlsruhe Institute of Technology
%% Institute for Program Structures and Data Organization
%% Chair for Software Design and Quality (SDQ)
%%
%% Dr.-Ing. Erik Burger
%% burger@kit.edu
%%
%% Version 1.0.2, 2020-05-07

\section{Einleitung}
\label{ch:Introduction}

%% -------------------
%% | Example content |
%% -------------------


\subsection{Motivation}
Motivation

Neue Problemstellung, wozu braucht man Datenschutz und Privacy? Was ist das Ziel der Seminararbeit?


\subsection{Verwandte Themen und Hintergrund}
\label{sec:Introduction:Citation}

Einstiegsliteratur - 
\cite{Strubbe2017}
\cite{Kiometzis2017}

Begriff C-ITS - \cite{CITS2016}, andere Begriffe einführen.

Initiativen in Deutschland und EU, Geschichte und Entwicklung

Car-2-Car Communication Consortium - \cite{Car2Car}

Related work - Literatur, die sich mit der Fragenstellung beschäftigt - z.B. \cite{Jochum2020}
%A citation: \cite{becker2008a} For referencing, see \autoref{sec:Introduction:Figures}

%\subsection{Example: Figures}
%\label{sec:Introduction:Figures}
%\begin{figure}
%\centering
%\includegraphics[width=4cm]{images/sdqlogo}
%\caption{SDQ logo}
%\label{fig:sdqlogo}
%\end{figure}

%A reference: The SDQ logo is displayed in \autoref{fig:sdqlogo}. 
%(Use \code{\textbackslash autoref\{\}} for easy referencing.) 
%
%\subsection{Example: Tables}
%\label{sec:Introduction:Tables}
%\begin{table}
%\centering
%\begin{tabular}{r l}
%\toprule
%abc & def\\
%ghi & jkl\\
%\midrule
%123 & 456\\
%789 & 0AB\\
%\bottomrule
%\end{tabular}
%\caption{A table}
%\label{tab:atable}
%\end{table}

%% --------------------
%% | /Example content |
%% --------------------