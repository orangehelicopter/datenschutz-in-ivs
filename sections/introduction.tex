%% LaTeX2e class for seminar theses
%% sections/content.tex
%% 
%% Karlsruhe Institute of Technology
%% Institute for Program Structures and Data Organization
%% Chair for Software Design and Quality (SDQ)
%%
%% Dr.-Ing. Erik Burger
%% burger@kit.edu
%%
%% Version 1.0.2, 2020-05-07

\section{Einleitung}
\label{ch:Introduction}

%% -------------------
%% | Example content |
%% -------------------

Das 21. Jahrhundert bringt verschiedene Herausforderungen mit sich, die wir als Gesellschaft meistern müssen - Klimawandel, steigende Population, Umweltverschmutzung und viel mehr. Um sie zu bewältigen brauchen wir neue Technologien, die uns im Alltag unterstützen und dabei helfen, die individuelle Verantwortungslast zu reduzieren. Darum ist der effiziente und baldige Einsatz Intelligenter Verkehrssysteme ein wichtiger Bestandteil deutscher Verkehrspolitik. Im Mittelpunkt hierbei stehen intelligente Fahrzeuge und andere Straßenanlagen, die durch Kooperation miteinander den Straßenverkehr effizienter, sicherer und umweltfreundlicher gestalten\footnote{\emph{BMVI - Intelligente Verkehrssysteme im Straßenverkehr}, URL: \url{https://www.bmvi.de/SharedDocs/DE/Artikel/DG/ivs-im-strassenverkehr.htm}.}. \nocite{BMVI}

In naher Zukunft werden nach und nach im Alltag Fahrzeuge eingeführt, die miteinander (Car-to-Car, kurz C2C) und mit der Infrastruktur (Car-to-Infrastructure, kurz C2I) kommunizieren können. Der Oberbegriff zu dieser Art der Kommunikation lautet Car-to-X (C2X). Der Datenaustausch in solchen Intelligenten Verkehrssystemen (IVS) findet statt, um Unfälle auf den Straßen zu vermeiden und damit die Sicherheit im Straßenverkehr zu gewährleisten. Dafür senden Fahrzeuge gegenseitig verkehrsrelevante Daten, wie zum Beispiel Beschleunigung, Geschwindigkeit, Länge und Gewicht des Fahrzeugs. Diese Daten werden verwendet, um ein Verkehrslagebild zu erstellen und zu verteilen, damit teilnehmende Fahrzeuge stets die aktuellsten Informationen besitzen und auf eintretende Verkehrssituationen geeignet reagieren können. 

Diese neue Problemstellung bringt einige datenschutzrechtlichen Fragen mit sich, die einen geeigneten Rechtsrahmen benötigen. Es soll sowohl den technischen Schutzzielen der Informationssicherheit nachgekommen werden (Vertraulichkeit und Integrität), als auch den rechtlichen Vorschriften, z.B. der Datenschutz-Grundverordnung (DSGVO). Das Ziel von dieser Seminararbeit ist es, die datenschutzrechtliche Relevanz der C2X-Kommunikation zu untersuchen. Dafür wird zuerst die technische Funktionsweise der C2X-Kommunikation erläutert, mit einem Schwerpunkt auf Nachrichtenformaten und Public-Key Kryptografie. Außerdem wird die Public-Key Infrastruktur in Europa beschrieben, die den Betrieb der IVS praktisch ermöglicht. 

Anschließend wird auf die Herstellbarkeit des Personenbezugs eingegangen, da sie entscheidend für die Wirkung der DSGVO ist. Es werden Möglichkeiten diskutiert, wie man aus den grundsätzlich nicht-personenbezogenen Daten, die in einer IVS Infrastruktur generiert und gesammelt werden, das Fahrzeug und ggf. den Fahrzeughalter identifizieren kann. Im Zusammenhang damit werden technische Möglichkeiten für die Generierung der individuellen Bewegungs- und Verhaltensprofile erläutert. Im Anschluss wird die Vereinbarkeit mit datenschutzrechtlichen Prinzipien diskutiert und es werden Datenschutzmaßnahmen vorgeschlagen, die das Nachkommen der gesetzlichen Pflichten ermöglichen.


\section{Hintergrund}
\label{ch:Background}

Bereits seit etwa Mitte der 2000er existieren einige technischen Lösungen für die Nahbereichskommunikation zwischen Fahrzeugen und ggf. Infrastrukturkomponenten. Zum Beispiel, die IEEE 802.11p ist eine WLAN-Variante, die aktuell in Europa als Funkstandard für Intelligente Verkehrssysteme (IVS) gilt. Weiterhin existieren mehrere Spezifikationen des ETSI (Europäisches Institut für Telekommunikationsnormen), die unter anderem Nachrichtenformate festlegen\footnote{\emph{Strubbe/Thenée/Wieschebrink}, DuD 41.4, S. 224}. \nocite{Strubbe2017}

Das Ziel der IVS ist es vor allem, den Verkehrsfluss zu verbessern und Unfälle vorzubeugen. Die Autofahrer sollen so früh wie möglich vor Gefahren auf der Fahrstrecke gewarnt werden, selbst wenn diese noch nicht im Sichtbereich sind. So können sie entsprechend reagieren und ihr Fahrverhalten frühzeitig anpassen, um Staus oder sogar zusätzliche Unfälle zu vermeiden. Als ein Beispiel im IVS Kontext kann man Sonderfahrzeuge nehmen (z.B. Kranken- oder Feuerwehrwagen) - diese sollten eine sogenannte ``Blaulichtnachricht''  versenden und somit andere Fahrzeuge darauf hinweisen, dass sie eine Rettungsgasse bilden sollten. Die C2X-Nachrichten auf Basis von 802.11p haben eine Reichweite von bis zu 800 Metern, was herkömmliche Sirenen von z.B. Krankenwagen weit übersteigt, besonders in dicht bewohnten Stadtgebieten.

Die EU hat bereits vor zehn Jahren eine Rechtsgrundlage für intelligente Verkehrssysteme geschaffen. Es handelt sich um die RL 2010/40/EU zum Rahmen für die Einführung intelligenter Verkehrssysteme im Straßenverkehr und für deren Schnittstellen zu anderen Verkehrsträgern (IVS-RL)\footnote{\emph{Jochum}, ZD 2020, S. 497.}. Im November 2016 wurde von der Europäischen Kommission eine Strategie für Kooperative Intelligente Verkehrssysteme geschaffen\footnote{\emph{Eine europäische Strategie für Kooperative Intelligente Verkehrssysteme - ein Meilenstein auf dem Weg zu einer kooperativen, vernetzten und automatisierten Mobilität}, 2016.}\nocite{CITS2016}, mit der ein praktischer Rahmen für eine Einführung von IVS in Europa definiert wurde. 

Die Initiativen zur Standardisierung der C2X-Kommunikation liegen jedoch weiter in der Vergangenheit. Bereits in 2002 wurde das Car-2-Car Communication Consortium gegründet (C2C-CC)\footnote{\emph{CAR 2 CAR Communication Consortium}, URL: \url{https://www.car-2-car.org}.}\nocite{Car2Car}, dem die meisten Fahrzeughersteller und große Zulieferer beigetreten sind. Das Ziel davon ist es, die C2C-Kommunikation zu standardisieren und den Rollout im europäischen Markt voranzutreiben. Im Jahr 2019 wurde ein wichtiger Meilenstein auf diesem Weg erreicht, da die ersten Fahrzeuge mit kooperativer C2X für den Markt frei verfügbar gemacht wurden. Hier geht es um sogenanntes Day-1 Enrolment, den ersten Schritt auf dem Fahrplan von IVS. Die drei Phasen davon werden in der Tabelle \ref{table:1} näher beschrieben.

\begin{table}[h!]
	\centering
	\begin{tabular}{|m{4em}|m{10em}|m{10em}|m{8em}|} 
		\hline
		& Day 1 & Day 2 & Day 3 \\ [1ex]
		\hline
		Fokus & Awareness driving & Sensing driving & Cooperative driving \\ [1ex]
		\hline 
		Beispiele & \begin{itemize}[leftmargin=0.3cm] \item Warnung vor langsamen oder stehenden Fahrzeugen
		
		\item Warnung vor Straßenarbeiten
		
		\item Warnung vor sich nähernden Einsatzfahrzeugen
		
		\item Anzeige von Verkehrszeichen im Fahrzeug
		
		\item Anzeige Missachtung von Verkehrsampeln / Sicherheit auf Kreuzungen 
		
	\end{itemize} & 
	\begin{itemize}[leftmargin=0.3cm] \item Überholwarnung
	
	\item Erweiterte Sicherheit auf Kreuzungen 
	
	\item Kooperative adaptive Geschwindigkeitsregelung
	
	\item Warnung vor langfristigen Straßenarbeiten
	
	\item Spezielle Fahrzeugpriorisierung
	
	\end{itemize} & 
	\begin{itemize}[leftmargin=0.3cm] \item Kooperatives Überholen
	
	\item Kooperativer Spurwechsel
	
	\item Platooning\footnotemark
	
	\end{itemize} \\ 
		\hline
	\end{tabular}
	\caption[Phasen der Einführung von IVS-Diensten \nocite{Car2Car}]{Phasen der Einführung von IVS-Diensten übersetzt nach C2C-CC \footnotemark}
	\label{table:1}
\end{table}

\footnotetext{Unter Platooning (\textit{dt.} ``elektronische Deichsel'') versteht man ein Steuerungssystem für den Straßenverkehr, bei dem mehrere vernetzte Fahrzeuge mit Hilfe einer adaptiven Geschwindigkeitsregelung in sehr geringem Abstand hintereinander fahren können.}

\footnotetext{\emph{CAR 2 CAR Communication Consortium}, URL: \url{https://www.car-2-car.org}.}\nocite{Car2Car}

Auf Ersuchen der Europäischen Kommission wurde eine Studie durchgeführt, in welcher die Kosten und der Nutzen der durch IVS unterstützten Dienste für den Straßenverkehr in den Mitgliedstaaten untersucht wurden. Im Ergebnis wurde erwiesen, dass bei einer europaweiten Einführung von Day 1 IVS-Diensten im Zeitraum 2018 bis 2030 das Nutzen-Kosten-Verhältnis bis zu 3:1 betragen wird (falls die Interoperabilität zwischen den Mitgliedsstaaten sichergestellt wird). Dies bedeutet, dass jeder in die für den Day 1 vereinbarten IVS-Dienste investierte Euro einen Nutzen von bis zu drei Euro generieren dürfte\footnote{\emph{Study on the Deployment of C-ITS in Europe: Final Report. Framework Contract on Impact Assessment and Evaluation Studies in the Field of Transport MOVE/A3/119-2013-Lot No 5 Horizontal}, Techn. Ber. MOVE/C.3./No 2014- 794. 2016.}. \nocite{StudyDeployment} 

Wie oben erwähnt, wurden die Day 1 Dienste bereits eingesetzt und sind im europäischen Markt verfügbar. Der Einsatz von Day 2 und Day 3 Anwendungen hingegen befindet sich in der Forschungsphase, die ebenfalls vom Car-2-Car Communication Consortium (C2C-CC) vorangetrieben wird. Ein vom C2C-CC definierter Meilenstein ist eine Marktpenetration von 3\%-5\%, da ab dann die ersten Vorteile von kooperativem C2X von den Nutzern gespürt werden können\footnote{\emph{CAR 2 CAR Communication Consortium FAQs}, URL: \url{https://www.car-2-car.org/about-c-its/c-its-faqs/}.}\nocite{Car2Car}. Daher verfolgt C2C-CC folgende Ziele, um diese Entwicklung voranzutreiben:

\begin{itemize}
	\item Interoperabilität und grenzübergreifende Nutzung von Car-to-Car-Systemen
	\item Entwicklung der realistischen Strategien und Geschäftsrahmen für die Einführung von C2X
	\item Kooperation mit der Straßeninfrastruktur zur Entwicklung und Bereitstellung von C2I in der Automobilindustrie
	\item  Zuweisung eines gebührenfreien europaweiten exklusiven 5,9-GHz-IVS-Frequenz-bandes für kooperative C2X-Anwendungen
	\item Weltweite Standardisierung von kooperativen C2X Systemen, insbesondere in Kooperation mit ETSI TC ITS
\end{itemize}

%A citation: \cite{becker2008a} For referencing, see \autoref{sec:Introduction:Figures}

%\subsection{Example: Figures}
%\label{sec:Introduction:Figures}
%\begin{figure}
%\centering
%\includegraphics[width=4cm]{images/sdqlogo}
%\caption{SDQ logo}
%\label{fig:sdqlogo}
%\end{figure}

%A reference: The SDQ logo is displayed in \autoref{fig:sdqlogo}. 
%(Use \code{\textbackslash autoref\{\}} for easy referencing.) 
%
%\subsection{Example: Tables}
%\label{sec:Introduction:Tables}
%\begin{table}
%\centering
%\begin{tabular}{r l}
%\toprule
%abc & def\\
%ghi & jkl\\
%\midrule
%123 & 456\\
%789 & 0AB\\
%\bottomrule
%\end{tabular}
%\caption{A table}
%\label{tab:atable}
%\end{table}

%% --------------------
%% | /Example content |
%% --------------------