%% LaTeX2e class for seminar theses
%% sections/content.tex
%% 
%% Karlsruhe Institute of Technology
%% Institute for Program Structures and Data Organization
%% Chair for Software Design and Quality (SDQ)
%%
%% Dr.-Ing. Erik Burger
%% burger@kit.edu
%%
%% Version 1.0.2, 2020-05-07

\section{Technische Grundlagen}
\label{ch:FirstContentSection}

%% -------------------
%% | Example content |
%% -------------------

Hier grob den technischen Ablauf beschreiben und die darauf folgenden Themen kurz vorstellen.

\subsection{Public-Key Infrastruktur}
\label{sec:FirstContentSection:FirstSubSection}
Im Allgemeinen
In Europa - Trust List Managers

\cite{SecurityCITS} - European certificate policy for C-ITS

PKI für V2X Kommunikation - Aufbau EA, AA, EE - \cite{Strubbe2017}
Verschlüsselung, Trust and Privacy Management - \cite{ETSI2018}

ECDSA Algorithmus basierend auf elliptischen Kurven (ECC) - \cite{Barker2013}

\subsection{Nachrichtenformate}
\label{sec:FirstContentSection:SecondSubSection}

\cite{ETSI2013} - Security Headers and Certificate Formats

Eventuell ein Schaubild für die Nachrichtenformate erstellen und hier referenzieren

\subsection{Angriffsmöglichkeiten}
\label{sec:FirstContentSection:ThirdSubSection}

"Big Brother Angreifer" \cite{Wiedersheim2010}

Überwachungstool - CAM-Trace

Secondary Vehicle Identifier - \cite{Ullmann2016}

z:B: nervöse Fahrer können von ruhigen unterschieden werden - \cite{Dettki2005}

Andere Papers und andere möglichen Datenschutzverletzungen finden. 

\section{Vereinbarkeit mit datenschutzrechtlichen Prinzipien}
\label{ch:SecondContentSection}

Warum stellen IVS ein datenschutzrechtliches Problem dar? Ist DSGVO anwendbar? 

(Beispiel von rechtlicher Probl	emstellung und Richtlinien zu deren Lösung - \cite{EUCooperativeV2X} ) - wird wahrscheinlich nicht behandelt oder kurz erwähnt, da wir uns auf DSGVO konzentrieren

\subsection{Anwendbarkeit der DSGVO}
\label{sec:SecondContentSection:SecondSubsection}

Art.6 Abs.1c DSGVO - Personenbezug Fahrdaten. Sind Fahrdaten personenbezogen, btw. können sie für eine eindeutige Identifizierung der Person benutzt werden?

\cite{Weichert2016} - Datenschutzrechtliche Analyse der Car-2-Car-Communication, guter Ausgangspunkt.

\subsection{Grundsatz der Datenminimierung}

Art.5 Abs.1c DSGVO - "Datenminimierung". 

Geeignete Pseudonymisierung

\subsection{Recht auf Datenübertragbarkeit}

Art.20 DSGVO - "Recht auf Datenübertragbarkeit"

\cite{Straub2018} - Analyse der Datenübertragbarkeit, Begründung der DSGVO-Anwendbarkeit

\subsection{Andere Datenschutzmaßnahmen}

\cite{Seewald2018} - alle o.g. Datenschutzmaßnahmen + "obligation to raport a data breach, to conduct a data protection impact assessment". "analytical progress of AI". Guter Ausgangspunkt für Kapitel \ref{sec:SecondContentSection:SecondSubsection}

Die in \cite{Kiometzis2017} aufgeführten Empfehlungen untersuchen und analysieren. 

%Add additional content sections if required by adding new .tex files in the
%\code{sections/} directory and adding an appropriate 
%\code{\textbackslash input} statement in \code{thesis.tex}. 
%%% ---------------------
%% | / Example content |
%% ---------------------